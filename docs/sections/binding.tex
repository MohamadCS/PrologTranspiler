\section{Binding}

Binding is used for matching a variable, or a tuple of variables to
another expression. It does matching automatically, that is, if the
expression is an arithmitic term, it evalutes the term, and
matches the variable with the value of the term (like \texttt{is}
operator) otherwise it acts like \texttt{=} operator.

It have the intuitive syntax \texttt{Var <- Expr}. 

\subsection{Tuple unpacking}

For convinence, the language allows you to unpack the tuple such that it
entries match with named variables, using the syntax \texttt{TupleOfVar <-
Expr} when \texttt{Expr} is evaluted to a tuple with the same size as the tuple
of variables.

\subsection{Examples}

\begin{itemize}
	\item \texttt{X <- 3 * Y;}, evalues \texttt{3*Y} then matches it with \texttt{X}.
	\item \texttt{(X,Y) <- getMinMax(List);}, evalues \texttt{getMinMax(List)} to \texttt{tuple(Min,Max)} then
	      matches \texttt{X} with \texttt{Min} and \texttt{Max} with \texttt{Y}.
\end{itemize}

