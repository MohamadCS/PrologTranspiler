\section{Testing}

\texttt{Prolog*} provides a builtin testing features.

A test function is defined like a regular function, but it has no
name and no arguments, instead, a describtion string must be provided. 

\subsection{Syntax}

\begin{lstlisting}[language = Prolog]
test '<string of test desc>' tuple .
\end{lstlisting}

\textbf{Example}

\begin{lstlisting}[language = Prolog]
test 'HeapSort test on random list' (

    RandomList <- std:ForEach( 
                              std:MakeList(100,0), 
                              (X) => (std:RandomNum(1,100))
    );

    testing:EXPECT_EQ(bin_heap:HeapSort(RandomList),
                      std:SortList(RandomList)
    );
)
.
\end{lstlisting}

If at least one test is defined, \texttt{Prolog*} will define a new function called
\texttt{RunTests/0}, which will run all tests, in the order of their definitions. 

