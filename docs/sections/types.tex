\section{Types - deprecated}

\begin{note}
    This feature is deprecated for now, it needs a reimplementation.
\end{note}

Types can be used for type checking, that is, inforcing the user
to use a pass a specific type to the function.

Function arguments and binding to a variable can declare the variable's type 
using the syntax \texttt{Var : type}, if the type is a part of a module
then it must declare the module name before the type \texttt{Var : std:type}.

If the type is nullable then we pair it with \texttt{?}.

Example

\begin{lstlisting}[language = Prolog]

    import {
        'stdlib'
    }

    Sort(List : std:list, Func) :: (
        ...
    )
    .

    Main() :: (
        Sort(1, _); // Type mismatch.
        Sort([], Min); // ok.

        X : std:list <- 1; // Type mismatch.
    )
    .

\end{lstlisting}

\subsection{Syntax}

We can define a type by 

\begin{lstlisting}[language = Prolog]
    type node :: (number , node?, node?).


    Foo(Node : node) :: (

    )
    .

    Main() :: (
        Foo(node(100,1,1)); // Type mismatch
        Foo(node(100,nil,nil)); // ok
        Foo(node(100,node(100,nil,nil),nil)); // ok
    )
    .

\end{lstlisting}


