\section{Conditionals}

There are two types of conditionals if statements, and if-else
expressions.

If statements, evaluates the tuple in the if body if the condition is
true, while if-else evalutes one of them according to the condition.

The reason that \texttt{if} is a statement is that it must be always
vanishing to avoid ambiguity, since if the condition is false the if
must not evalute to a value, but a tuple entry must be known if its a
vanishing or non-vanishing at compile time.

In the otherhand, if-else provide us with a way to tell if the
expression is vanishing or not.

We can add some bindings at the head of the if condition in order for
a more compact syntax, this is an optional.

In order to allow for more readable nested if-else statements, if the
tuple has one entry only, and its non-vanishing, then we can remove
the parentheses.

\subsection{Syntax}
\begin{lstlisting}[language =Prolog]
    if  optional(IfHead |)  Condition then 
        tuple
\end{lstlisting}

\begin{lstlisting}[language =Prolog]
    if optional(IfHead |) Condition then 
        tuple
    else 
        tuple
\end{lstlisting}

\textbf{Example}

\begin{enumerate}
	\begin{lstlisting}[language =Prolog]
    if Idx = 0 then (
        match List {
            [] => [],
            [L | Ls] => [NewVal | Ls]
        }
    ) else (
        List <- [L | Ls]; 
        [L | Replace(Ls,Idx - 1, NewVal)]
    )
    \end{lstlisting}
\end{enumerate}

