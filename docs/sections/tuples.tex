\section{Tuples}

Tuples are simply lists of 
\begin{itemize}
    \item Expressions
    \item Bindings.
    \item If statments (Without else).
\end{itemize}

\noindent Tuples follows the following rules
\begin{itemize}
    \item If the expression is paired with \texttt{,} then the expression is 
        non-vanishing, that is its value is evaluated, and an entry in the final tuple.

    \item If the expression is paired with \texttt{;} then the expression is 
        vanishing, that is its value is evaluated, and its not an entry in the final tuple.

    \item Last item can be without a pair, in this case its interpreted as a non-vanishing entry. 

    \item If the tuple is empty then its a vanishing entry regardless of its pair.

    \item If the tuple entry is a direct predicate call, then its acts like in regular predicates, that is,
        if it evalutes to true then the execution of the function continues, otherwise it stopps. Regardless
        of the result, the entry would always be vanishing.
\end{itemize}

Tuples are compiled to the predicate \texttt{tuple(...)} when \texttt{...} are the non-vanishing
entries of the tuple.


\subsection{Examples}
\begin{enumerate}
    \item \texttt{(1,2)}.
    \item \texttt{([3,4],Max(4,3),(1,2))}, evalutes to \texttt{tuple([3,4],Max(4,3),(1,2))}.
    \item \texttt{([3,4];Max(4,3),(1,2);)}, evalutes to \texttt{tuple(Max(4,3),(1,2))}.
\end{enumerate}


\begin{note}
    The number of entries in the tuple's predicate is determined at compile time, for this reason,
    an if statement (Without else) can only be a vanishing statement.
\end{note}


