\section{Functions}


A functions, takes arguments, and returns a result. 

A function result is a tuple of expressions, unless
the tuple has one entry, then its the expresssion 
that the tuple contains.

The functions name must be a valid variable's name.

\subsection{Syntax}


\begin{lstlisting}[language = Prolog]
    Func(Arg1, Arg2, ..., ArgN) :: (
        Expr1,
        ...
        ExprN
    )
    .
\end{lstlisting}

\subsection{Arguments Alias}

We can refer to the i-th argument of the function with
the synatx \texttt{#i}.

\texttt{i} must be within the range of the function's arguments number.

We can also refer to the tuple that contains all the function's 
arguments using \texttt{#}.

Example:

\begin{lstlisting}[language = Prolog]
    Max(X,Y) :: (if #1 >= #2 then #1 else #2).
\end{lstlisting}



\subsection{Lambdas}

Lambdas are annonymos functions, and they are considered as 
expression, meaning they can be binded to variables, returned 
from functions, passed as an argument ...

They have the following syntax


\begin{lstlisting}[language = Prolog]
    (Arg1, Arg2, ..., ArgN) => (stmts)
\end{lstlisting}

Example

\begin{lstlisting}[language = Prolog]
    Foo(X,Y) :: (
        Max <- (X,Y) => (if #1 >= #2 then #1 else #2)
        Max(X,Y)
    )
    .
\end{lstlisting}

Note that \texttt{X,Y} are local to the lambdas.
