\section{Functions}


Functions takes arguments and returns a result.

A function's result is a tuple, unless the tuple has one entry, then
the function returns the expression that the tuple contain's.

A function's name must be a valid variable's name that starts with a capital
letter. 

The function will compile to a regular \texttt{Prolog} predicate but
with the first letter being a small letter, and with an extra
argument as the last arguments that stores the function's result.

\subsection{Syntax}

\begin{lstlisting}[language = Prolog]
Func(Arg1, Arg2, ..., ArgN) :: tuple .
\end{lstlisting}


\noindent \textbf{Example}

\begin{lstlisting}[language = Prolog]
Replace(List,Idx,NewVal) :: (

    if ListSize <- std:Size(List) | Idx < 0 ; Idx > ListSize - 1 then (
        write('Wrong Idx');
        Exit();
    );

    if Idx = 0 then (
        match List {
            [] => [],
            [L | Ls] => [NewVal | Ls]
        }
    ) else (
        List <- [L | Ls]; 
        [L | Replace(Ls,Idx - 1, NewVal)]
    )
)
.
\end{lstlisting}



\subsection{Arguments Alias}

We can refer to the i-th argument of the function with
the synatx \texttt{#i}.

\texttt{i} must be within the range of the function's arguments number.

We can also refer to the tuple that contains all the function's 
arguments using \texttt{#}.

Example:

\begin{lstlisting}[language = Prolog]
    Max(X,Y) :: (if #1 >= #2 then #1 else #2).
\end{lstlisting}

\begin{lstlisting}[language = Prolog]
    F(X) :: (if tuple(X) = # then 1 else 0). // Evaluates to 1
\end{lstlisting}



\subsection{Lambdas}

Lambdas are annonymos functions, and they are considered as 
expression, meaning they can be binded to variables, returned 
from functions, passed as an argument ...

\subsection{Syntax}
\begin{lstlisting}[language = Prolog]
    (Arg1, Arg2, ..., ArgN) => (stmts)
\end{lstlisting}

\subsection{Example}

\begin{lstlisting}[language = Prolog]
    Foo(X,Y) :: (
        Max <- (X,Y) => (if #1 >= #2 then #1 else #2)
        Max(X,Y)
    )
    .
\end{lstlisting}

Note that \texttt{X,Y} are local to the lambdas.

\section{The Main Function}

If the special function \texttt{Main/0} is defined, then prolog generates
a predicate with no arguments that defines this function, and it generates 
a directive that runs the function automatically at the end of the file. 


\begin{lstlisting}[language = Prolog]
    Main() :: (
        std:PrintLn("Hello");
    )
    .
\end{lstlisting}

In this example, if we load the file that contains this function, it will print \texttt{Hello} without
the need to call the predicate \texttt{main}.
