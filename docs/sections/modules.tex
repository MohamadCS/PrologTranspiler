\section{Modules}

Modules are an abstraction of the current modules system of Prolog, they
contain functions or types, that can be marked as public using the keyword
\texttt{pub} for other files to use when the import the module

\subsection{Syntax}

We can define a module using
\begin{lstlisting}[language =Prolog]
    module my_module {
        pub Foo() :: (...).
        pub type :: (...).
    }
\end{lstlisting}

And we can import the file that contains the module by listing the file name
without extenstion

\begin{lstlisting}[language =Prolog]
    import {
        'stdlib',
        'testlib',
        'heap',
    }
\end{lstlisting}

If we want to use a function from a module we've imported, we must declare 
the module name before the function's name.


\begin{lstlisting}[language =Prolog]
    import {
        'm1', // Has Foo/0 that prints 'hello'
        'm2'  // Has Foo/0 that prints 'world'
    }


    Main() :: (
        m1:Foo(); // Prints 'hello'
        m2:Foo(); // Prints 'world'
    )
    .

\end{lstlisting}





